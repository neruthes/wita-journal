\setentryid{004}
\setprimarytag{Enterprise}
\stdarticle{Navigating the AI Geoscape: A Multi-Faceted Analysis of Manus AI's Strategic Retreat from Mainland China}{%
	\authorrow{Neruthes}{}\\%
	\authorrow{Gemini}{(Google)}%
}{2025-07-14}

\articlecopyrightinfo{Public Domain}



\section*{Abstract}
This report analyzes the recent strategic pivot of Manus AI, a prominent Chinese AI agent startup, involving the cessation of its mainland China operations and the relocation of its global headquarters to Singapore.
While geopolitical tensions, particularly US investment restrictions and export controls on advanced AI chips, are widely cited as primary drivers, this analysis argues that Manus AI's move was also significantly influenced by intense domestic market competition, challenges in product differentiation, and a proactive global talent strategy.
By examining the interplay of these external and internal pressures, the report offers a holistic perspective on the complex decision-making processes of Chinese AI firms navigating an increasingly polarized global technology landscape, highlighting Singapore's role as a strategic ``third path'' hub.

\section{Introduction}

Manus AI, developed by Beijing Butterfly Effect Technology, emerged onto the global artificial intelligence (AI) scene in March 2025 with an invite-only AI agent that quickly garnered attention \cite{kiledjian_manus_promises}.
Its core proposition marked a significant departure from conventional conversational AI, as it was designed to autonomously perform complex, multi-step tasks such such as filtering resumes, analyzing stocks, and even writing and deploying code \cite{baytech_manus_guide}.
This capability positioned Manus AI as a general-purpose agent aiming to bridge the gap between human intention and tangible action, functioning as a self-directed digital assistant.
The platform generated considerable market buzz, with activation codes reportedly selling for as much as 100,000 yuan on secondary markets, underscoring the initial excitement surrounding its capabilities.

However, by July 2025, Manus AI undertook a significant, largely silent, operational adjustment, ceasing most of its activities in mainland China and relocating its global headquarters to Singapore \cite{techinasia_manus_singapore}.
This strategic shift involved substantial workforce reductions in China, with the majority of its approximately 120 staff laid off, while a core group of around 40 technical personnel were transferred to the new Singapore headquarters \cite{youtube_manus_withdrawal}.
Concurrently, Manus AI's digital footprint in China was erased, with official Chinese social media accounts (Weibo, Xiaohongshu) cleared of content and its official website displaying a message stating ``Manus is not available in your region'' for Chinese users \cite{aibase_manus_changes}.
This marked a stark change from its previous message indicating a ``Chinese version is under development''.

This report posits that Manus AI's strategic pivot was not merely a reaction to escalating US-China geopolitical tensions but a calculated, multi-faceted response driven by a complex interplay of external pressures, including US investment restrictions and chip export controls, and significant internal market dynamics, such as hyper-competition, declining user engagement, and a proactive global talent strategy.

\section{The Strategic Pivot: Confirmation and Context}

The operational adjustments undertaken by Manus AI in mid-2025 demonstrate a decisive move away from its original base in mainland China.
The company officially moved its headquarters from China to Singapore in June 2025 \cite{decoder_manus_geopolitical, onlinecitizen_manus_relocates, youtube_manus_withdrawal, cna_manus_relocates, citynews_manus_relocates}.
This relocation was publicly confirmed by co-founder and chief product officer Zhang Tao at the SuperAI conference in Singapore on June 18, 2025, where he stated that Singapore was now Manus AI's main base.
Beyond Singapore, the company also established offices in Tokyo and California, specifically San Mateo.

A significant component of this pivot was the restructuring of its workforce.
Beijing Butterfly Effect Technology, the Chinese operating entity behind Manus AI, laid off most of its approximately 120 employees in China, with those remaining receiving severance packages.
Crucially, approximately 40 core technical personnel were transferred to the new Singapore headquarters, indicating a strategic effort to retain essential expertise while shifting the operational center of gravity.

The company's digital presence also underwent a dramatic transformation.
Manus AI's official accounts on prominent Chinese social media platforms, Weibo and Xiaohongshu, were cleared of all content.
Furthermore, a previously announced partnership with Alibaba Cloud's chatbot, Tongyi Qianwen, was deleted, and a former employee confirmed the collaboration would not proceed.
Perhaps most indicative of the shift, the official Manus AI website, which once promised a ``Chinese version is under development,'' now displays ``Manus is not available in your region'' for users attempting to access it from China.

In its official communications, Manus AI stated that these adjustments were ``based on the company's own operating efficiency considerations'' and aimed to ``continue to focus on core business development and improve overall operational efficiency'' \cite{standard_manus_restructures}.

The comprehensive nature of these changes, particularly the clearing of Chinese social media accounts and the explicit message blocking access for Chinese users, points to a deliberate strategy to reshape the brand's identity.
This goes beyond mere operational restructuring; it suggests a conscious effort to re-brand Manus AI as a global, non-Chinese entity.
The objective appears to be to appeal to international investors and markets, especially in the US, where affiliations with China are increasingly perceived as a liability.
This strategic decision to shed its Chinese identity is a clear attempt to mitigate perceived geopolitical risks and enhance its standing in Western markets.

Moreover, the company's official statement about ``operational efficiency'' serves as a broad, corporate-friendly explanation for a complex and abrupt withdrawal.
While efficiency is a legitimate business objective, the scale and suddenness of Manus AI's pivot, combined with the explicit geopolitical and market pressures detailed in other reports, suggest that ``operational efficiency'' functions as a euphemism.
This allows the company to depoliticize its actions and avoid publicly acknowledging the full extent of external regulatory pressures and internal market struggles, thereby maintaining a neutral business narrative in a sensitive environment.

\begin{table}[htbp]
	\centering
	\caption{Key Events and Dates in Manus AI's China Operations Adjustment}
	\begin{tabularx}\linewidth{lX}
		\toprule
		\textbf{Date}   & \textbf{Event Description} \\
		\midrule
		March 2025      & Manus AI global debut/launch                                                \\
		March       & Manus AI monthly active users peak at 20 million                            \\
		April       & Benchmark's US\$75 million funding round, valuing Manus at US\$500 million  \\
		May         & Founders relocate to Singapore                                              \\
		May         & Manus AI monthly active users fall to 10 million                            \\
		May         & US Treasury Department reportedly reviewing Benchmark funding               \\
		June 18    & Co-founder Zhang Tao confirms Singapore as HQ at SuperAI conference         \\
		Mid-June    & Manus ads begin appearing in Singapore                                      \\
		July 9-11  & News reports confirm HQ shift, China job cuts, and social media changes     \\
		\bottomrule
	\end{tabularx}
\end{table}

\section{Geopolitical Pressures: The Dominant Narrative}

The most frequently cited catalysts for Manus AI's relocation are the escalating geopolitical tensions between the United States and China, particularly concerning advanced technology.
These external pressures have created a challenging operating environment for Chinese AI firms seeking global reach.

The Biden Administration's outbound investment regulations, implemented in January 2025, represent a significant constraint on US investors interested in advanced Chinese technology companies.
These rules specifically target US investments in Chinese entities involved in sensitive technologies, including AI, by either prohibiting certain transactions or requiring notification to the US Treasury Department.
This legislative framework imposes substantial compliance risks and heightened due diligence requirements for US investors, including private equity and venture capital funds.
Manus AI's US\$75 million funding round in April 2025, led by the Silicon Valley venture capital firm Benchmark, immediately drew scrutiny from the US Treasury Department, which initiated a review to determine its compliance with these new restrictions.
This demonstrates how investment capital itself has become a tool in geopolitical competition, with the US government actively seeking to prevent its capital from strengthening Chinese AI capabilities perceived as national security risks.
The need for external capital for rapid growth in the AI sector is critical, and by relocating, Manus AI attempts to de-risk future fundraising efforts from American investors, illustrating how geopolitical considerations directly influence funding structures and corporate geography.

Complementing investment restrictions are the stringent US export controls on advanced AI chips.
Imposed in April 2025, these controls ban the sale of high-end AI chips, such as Nvidia's H100, to China \cite{cna_manus_relocates}.
Earlier restrictions in October 2022 and 2023 had already targeted Nvidia's A100 and H100, and even customized lower-performance variants like the H800 and A800.
These advanced chips are indispensable for training the complex algorithms that power general-purpose AI agents like Manus AI.
Consequently, Chinese firms face considerable obstacles in acquiring these essential components.
Nvidia's CEO, Jensen Huang, has publicly criticized these controls, labeling them a ``failure'' for inadvertently spurring Chinese companies to accelerate their own AI development and noting a significant decline in Nvidia's market share in China \cite{timesofindia_nvidia_china_chip, ainvest_nvidia_balancing}.

In this challenging environment, Singapore has rapidly emerged as a strategically important hub for Chinese-origin technology companies seeking to navigate the tensions between Washington and Beijing.
The relocation to Singapore is explicitly aimed at mitigating the impact of US investment restrictions and the escalating US-China AI competition.
Singapore offers a compelling value proposition: better access to international markets, crucial computing resources, and global capital.
It also provides a pathway for companies to attract Western clients and investors while sidestepping potential restrictions that might otherwise be imposed on China-based entities.
This strategic maneuver is not unique to Manus AI; other Chinese tech giants, such as the fast fashion company Shein and the social media platform TikTok, have similarly emphasized their Singapore headquarters while maintaining production networks or control links in China.
This trend exemplifies a broader strategy for Chinese AI companies to establish a ``third path'' in the increasingly polarized US-China tech ecosystem.
Singapore, in this context, offers a neutral, internationally connected environment that allows these companies to maintain proximity to China's talent pool while presenting themselves as global entities, thereby bypassing direct US scrutiny and accessing critical inputs like advanced chips and venture capital.
This represents a strategic arbitrage of geopolitical friction.

\section{Beyond Geopolitics: Internal Market Dynamics and Strategic Repositioning}

While geopolitical pressures undeniably exerted significant influence, Manus AI's strategic pivot was also profoundly shaped by internal market dynamics and a proactive repositioning strategy.

\subsection{The ``War of a Hundred Models'': Intense Domestic Competition}

China's domestic AI market is characterized by an extraordinary level of competition, often referred to as the ``war of a hundred models.'' With over 130 large language models (LLMs) developed, China accounts for approximately 40\% of the global total.
This proliferation has led to an intensely crowded market, raising concerns about sustainability and the viability of numerous players.
The fierce competition has ignited a ``price war,'' with major Chinese tech giants like ByteDance, Alibaba, and Baidu drastically cutting prices on their LLM-based services in a bid to attract users.
This aggressive pricing environment makes it exceedingly difficult for smaller startups, like Manus AI, to establish viable business models and achieve profitability within the domestic arena.
The competitive landscape is further intensified by the entry of established tech giants, such as ByteDance with its Coze Space and Baidu with its AgentBuilder, which have introduced rival AI products, directly competing for market share within China's burgeoning AI sector.
This ``war of a hundred models'' is not merely a challenge; it acts as a powerful internal force pushing Chinese AI startups outwards.
While geopolitical factors create external barriers, the intense domestic competition, characterized by a proliferation of similar models and aggressive price wars, makes the Chinese market less attractive for sustainable growth, especially for smaller players.
This internal pressure to seek more favorable profit potential internationally complements and amplifies the external geopolitical push, making internationalization a dual imperative for survival and expansion.

\subsection{Product Viability and User Engagement Challenges}

Manus AI also faced significant challenges related to its product's viability and user engagement.
The company experienced a notable decline in its user base, with monthly active users plummeting from approximately 20 million in March to around 10 million by May 2025.
This decline suggests underlying issues beyond external pressures.

Despite initial hype, critics have argued that Manus AI lacks ``real technological breakthroughs,'' describing it as more of a ``shell'' that relies heavily on existing large models, such as Anthropic's Claude family and Alibaba's Qwen models, and pre-existing toolchains, rather than developing original core technology.
This reliance on third-party LLMs contributes to higher operational costs and raises questions about scalability.
Furthermore, early feedback on Manus AI highlighted technical issues, including reports of system instability, frequent crashes, inaccurate data generation, and slower processing speeds compared to some competitors.
Users also noted difficulties with straightforward operations and integration issues within its multi-agent system.
The critique that Manus AI acts as a ``shell'' relying on existing LLMs points to a deeper challenge in China's AI ecosystem beyond just quantity.
In a market saturated with LLMs, true differentiation requires novel technological advancements.
If Manus AI is perceived as lacking originality and merely integrating third-party models, its declining user numbers become understandable, as users may gravitate towards offerings from larger players with deeper research and development capabilities or more unique features.
This suggests that the domestic market's ``war of a hundred models'' is also a ``war of innovation,'' where only truly differentiated or highly efficient applications can thrive.
The relocation might also be an attempt to access a global talent pool and research and development environment more conducive to developing proprietary, breakthrough technology \cite{weforum_china_ai_breakthroughs}.

\subsection{Global Talent Strategy and Operational Efficiency}

Manus AI's pivot also reflects a deliberate global talent strategy and a drive for operational efficiency.
The company has aggressively begun recruiting new talent in Singapore, with job listings for positions such as data analyst and AI agent engineer, and is also hiring in the US and Japan.
This robust recruitment drive signifies an intention to build a strong international talent base.

Crucially, the strategic transfer of around 40 core technical personnel from China to Singapore ensures the retention of critical research and development capabilities while physically relocating them to a more favorable operating environment.
In the highly competitive AI talent market, especially for specialized expertise, retaining key engineers and data scientists is paramount.
By relocating them to Singapore, Manus AI can leverage Chinese engineering talent while positioning itself in a global hub that offers better access to international markets, computing resources, and potentially a less restrictive research and development environment.
This allows the company to maintain its technical backbone while shedding the geopolitical baggage associated with being fully China-based.
Manus AI's official statements about enhancing ``operational efficiency'' align with the need to streamline operations in a less competitive, more globally accessible environment.
The founder, Zhang Tao, also indicated considering a ``full separation of its China and international operations'', suggesting a long-term strategy to completely de-link the global Manus AI brand from its Chinese origins.

\section{Interplay of Factors: A Holistic Perspective}

Manus AI's strategic retreat from mainland China was not a singular response to one dominant factor but rather a complex, multi-pronged maneuver necessitated by the synergistic pressures of geopolitical constraints and internal market dynamics.
The external pressures, particularly the US outbound investment regulations implemented in January 2025, significantly limited Manus AI's access to crucial Western capital.
This made it increasingly challenging for the company to sustain its operations and compete effectively in the capital-intensive ``war of a hundred models'' within China.
Simultaneously, the US chip export controls, particularly those imposed in April 2025, directly impacted Manus AI's ability to acquire the advanced computing resources essential for training its sophisticated AI algorithms.
This fundamental requirement for an AI agent startup was severely hindered, further impeding its technological competitiveness in an already crowded domestic market.
The confluence of these external pressures with the internal market saturation and the company's declining user numbers created an untenable operating environment in mainland China, rendering the strategic pivot a matter of both survival and future growth.

Manus AI's actions illustrate the ``geopolitical tax'' that Chinese tech companies face, forcing them to incur significant costs, including layoffs, relocation, and brand re-orientation, simply to operate globally.
The comprehensive nature of Manus AI's withdrawal—encompassing headquarters relocation, mass layoffs, and the erasure of its digital footprint—and its explicit aim to reduce ``geopolitical risks'' and gain ``Western credibility'' suggest that remaining fully China-based imposed an unacceptable cost in terms of access to capital, markets, and technology.
This ``tax'' compels a strategic de-coupling of the product and brand from its country of origin, even if the parent company, Butterfly Effect, maintains a presence in China.
This results in a complex, bifurcated operational model.

The relocation to Singapore served as a multi-pronged strategy designed to address these intertwined challenges.
Firstly, it facilitates access to US venture capital that would otherwise face severe restrictions under the new regulations, as evidenced by the Benchmark funding and the subsequent US Treasury review.
Secondly, Singapore provides better access to international computing resources, including advanced chips, thereby circumventing the impact of US export controls.
Thirdly, the move offers refuge from China's ``war of a hundred models'' \cite{perplexity_china_hundred_models}, where the profit potential in international markets appears more favorable compared to the intensely competitive domestic arena.
Fourthly, establishing headquarters in Singapore strategically positions Manus AI as a global company, enhancing its appeal to Western clients and investors and helping it bypass potential sanctions.
Finally, the aggressive recruitment in Singapore and the strategic transfer of core Chinese technical staff allow Manus AI to optimize its talent pool by leveraging Chinese engineering expertise while operating from a more globally integrated base.

Singapore's role in this context is not merely as a financial or logistical hub but as an emerging ``neutral'' ground for AI innovation, specifically for companies caught in geopolitical crosscurrents.
The repeated mention of Singapore as a strategic base for Chinese-origin tech companies, including Shein, TikTok, HeyGen, and WIZ.AI, indicates that its value extends beyond geographical proximity to China.
Its stable regulatory environment, strong international ties, and access to global talent and infrastructure make it an attractive ``safe harbor'' where companies can pursue technological development and market expansion without the direct political scrutiny associated with either the US or China.
This positions Singapore as a critical enabler of a more fragmented, multi-polar AI ecosystem.

\begin{table}[htbp]
	\centering
	\caption{Drivers Behind Manus AI's Strategic Pivot: Geopolitical vs.
		Market/Internal Factors}\small
	\begin{tabularx}\linewidth{lXX}
		\toprule
		\textbf{Category}             & \textbf{Geopolitical Factors}                                                                                                                            & \textbf{Market/Internal Factors}                                                              \\
		\midrule
		\textbf{Investment Access}    & US Outbound Investment Regulations (Jan 2025): Prohibitions/notification for US investments in Chinese AI, creating compliance risks for US VCs.
		US Treasury Probe of Benchmark Funding: Scrutiny over US\$75M investment due to Chinese ties \cite{foxrothschild_us_investment, akingump_us_regulations}.
		                              &                                                                                                                                                                                                                                                          \\
		\addlinespace
		\textbf{Resource Access}      & US Chip Export Controls (April 2025): Restrictions on advanced AI chips (Nvidia H100) vital for training AI \cite{timesofindia_nvidia_china_chip}.
		                              &                                                                                                                                                                                                                                                          \\
		\addlinespace
		\textbf{Market Environment}   & Perception of ``Chinese connections as a risk'' in US market.
		                              & ``War of a Hundred Models'': Intense domestic competition with over 130 LLMs, leading to market saturation and price wars.
		Competition from Chinese Tech Giants: ByteDance (Coze Space), Baidu (AgentBuilder) introducing rival products.
		\\
		\addlinespace
		\textbf{Product Performance}  &                                                                                                                                                          & Declining User Numbers: Monthly active users fell from 20M (March) to 10M (May 2025).
		Critiques on Innovation: Perceived lack of ``real technological breakthroughs,'' reliance on third-party LLMs.
		\\
		\addlinespace
		\textbf{Operational Strategy} &                                                                                                                                                          & Operational Efficiency Goals: Company's stated reason for restructuring.
		Global Talent Strategy: Aggressive recruitment in Singapore, transfer of core technical staff.
		\\
		\bottomrule
	\end{tabularx}
\end{table}

\section{Conclusion}

Manus AI's strategic retreat from mainland China represents a complex and multi-faceted response to an increasingly challenging operating environment.
While geopolitical pressures, specifically US investment restrictions and export controls on advanced AI chips, undeniably played a significant role by limiting access to crucial capital and indispensable technology, internal market dynamics were equally influential.
The hyper-competitive ``war of a hundred models'' in China, characterized by market saturation and aggressive price wars, coupled with Manus AI's declining user numbers and critiques regarding its perceived lack of unique technological breakthroughs, created a compelling internal impetus for seeking international markets.

The relocation to Singapore therefore signifies a strategic maneuver designed to achieve several critical objectives simultaneously: securing access to vital funding, gaining access to essential computing resources, escaping the intense domestic market saturation, and strategically repositioning Manus AI as a global entity.
This approach allows the company to leverage its Chinese talent base while shedding the geopolitical liabilities associated with being fully China-based.

Manus AI's case serves as a microcosm of the broader challenges faced by Chinese AI startups navigating a bifurcated global technology landscape.
It highlights a growing trend of ``de-sinicization'' or the adoption of ``third path'' strategies, where companies proactively seek to establish operational and brand neutrality outside mainland China to access global markets and capital.
This trend suggests a potential fragmentation of the global AI ecosystem, with new innovation hubs emerging in geopolitically neutral territories.

The success of Manus AI's pivot will serve as a critical case study for other Chinese technology firms contemplating similar moves.
The ``third path'' strategy, while inherently costly and complex, may become an increasingly vital pathway for Chinese innovation to achieve global scale and secure necessary resources amidst ongoing geopolitical tensions.
Future developments will likely see more Chinese-origin companies adopting hybrid operational models, maintaining some research and development or production links to China while strategically relocating core business functions and brand identity to international hubs like Singapore.

\printbibliography


