\setentryid{001}
% \nocite{*}

\stdarticle{The founding of Weight in the Attention, a journal for the AIGC Zeitgeist}{%
	\authorrow{Neruthes}{WITA Editor}%
}{2025-07-13}




\section*{Abstract}
This article is a founding memo of the Weight in the Attention journal, a project inspired by the recent AIGC bubble.
In time of post-COVID economic stringency, the AIGC landscape presents a dramatic spectacle
of eager to recreate the good old pre-COVID entrepreneurship hype.
Existing works \footfullcite{agibar} have demonstrated a promising vision
for making fun of the time and the trend as both an outsider and an insider.
The author examined possible vacancies for AIGC-oriented humor and satire and proposed
the creation of this journal as a leisure platform.



\section{Zeal, FOMO, and Anxiety}
The introduction of DeepSeek in early 2025 put an extra burden \cite{tonypengds}
on already stressed LLM engineers.
Difficulties were observed on enjoying shared moments with friends who work at its competitors.
In a highly competitive environment with little regulatory determination on maintaining work-life-balance,
the fear of missing out (FOMO) could easily pass down from top executives all the way along to
frontline product managers and software engineers,
regardless of working on training models, tweaking contexts, or developing supportive systems.



\section{Buzzwords, Burnout, and Depression}
On the other hand, being out of the AIGC industry does not mean feeling better.
Since the release of ChatGPT, buzzwords proliferated one-after-another ---
transformer, decoder, attention, agents, context, RAG, etc.
The landscape looked like the radical evolution of web frontend during 2014-2020 after with radical abandon of jQuery ---
Backbone, Meteor, Vue, Angular, React, Gulp, Grunt, WebPack, Next.js, etc.
The quantity of technical stack choices in web frontend projects were doubling every 18 months \cite{Moore1965}.

Both trends may lead to the same consequence --- burnout.
I had personally experienced the burnout with the radical evolution of web frontend,
and later decided to stick to good old vanilla approach for small projects of personal use
and to wait till a promising silver bullet emerges to give the zeal a pause,
otherwise I would have no cognitive resource to keep learning new web frontend stuff
especially when it was not my main focus.
No person is absolutely immune to burnout and attention is not infinite.
To keep up to date does not constitute to follow every tiny footstep
without examining its prowess to remain part of best practices in subsequent years
given that one does not personally earn a living from it.
In hindsight, the moments suggested clues of depression.


\section{Content Orientation}
// TODO


\section{Conclusion}
// TODO



\printbibliography
